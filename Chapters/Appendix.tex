\setcounter{chapter}{0} 
\renewcommand{\thechapter}{\Alph{chapter}}

\chapter{Anhang}
\label{chap:anhang}

\section*{SI-Präfixe}
\begin{table}[ht]
\centering
\begin{tabular}{cccr}
\hline
\textbf{Präfix} & \textbf{Symbol} & \textbf{Potenzdarstellung} & \textbf{Dezimaldarstellung} \\
\hline
Yotta & Y & $10^{24}$ & 1'000'000'000'000'000'000'000'000 \\
Zetta & Z & $10^{21}$ & 1'000'000'000'000'000'000'000 \\
Exa & E & $10^{18}$ & 1'000'000'000'000'000'000 \\
Peta & P & $10^{15}$ & 1'000'000'000'000 \\
Tera & T & $10^{12}$ & 1'000'000'000 \\
Giga & G & $10^{9}$ & 1'000'000'000 \\
Mega & M & $10^{6}$ & 1'000'000 \\
Kilo & k & $10^{3}$ & 1'000 \\
Hekto & h & $10^{2}$ & 100 \\
\hline
\hline
Dezi & d & $10^{-1}$ & 0.1 \\
Centi & c & $10^{-2}$ & 0.01 \\
Milli & m & $10^{-3}$ & 0.001 \\
Micro & $\mathrm{\mu}$ & $10^{-6}$ & 0.000001 \\
Nano & n & $10^{-9}$ & 0.000000001 \\
Pico & p & $10^{-12}$ & 0.000000000001 \\
Femto & f & $10^{-15}$ & 0.000000000000001 \\
Atto & a & $10^{-18}$ & 0.000000000000000001 \\
Zepto & z & $10^{-21}$ & 0.000000000000000000001 \\
Yocto & y & $10^{-24}$ & 0.000000000000000000000001 \\
\hline
\end{tabular}
\end{table}